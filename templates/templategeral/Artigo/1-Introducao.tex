\tab iOwlT is a geolocalization sound system based on the nature of prey searches by an nocturne owl. Using the technique of multilateration of signals, widely used in telecommunications, and the phase shift of a signal detected by distinct sensors well distributed in space, it is possible, with the knowledge of algebra and physics, to prove that in an $ N $-dimensional space, just $ N + 1 $ detectors are needed to accurately determine the origin of the event. As the real life has $ 3 $ dimensions only $4$ sound sensors determines the location of an event.

\tab Combining the power of the parallel processing, achieved with the use of the FPGA from DE-10 Nano board to deal with the appropriate simultaneous treatment of the audio signals by, mainly, adaptive digital filters (they adapt to the sound signals obtained in order to optimize their processing), and the use of machine learning algorithms trained to recognize the desired event, the present project aims to design an embedded system that could be coupled to a vehicle, which detects the location of gun-shooting events, that when identified will be displayed in a mobile application.

\tab It can be used in urban areas to detect sources of gun shots and even in forbidden hunting areas to identify possible hunters, the problem to be solved has applications that do not restrict the location of the source and of a specific sound, it can be adapted to the recognition of another audible pattern.