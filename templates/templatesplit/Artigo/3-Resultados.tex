O conjunto de dados MNIST foi separado utilizando o \textit{Cross-Validation 10-Folds}, portanto tem-se $54024$ imagens para treino, $6003$ imagens para validação e $10019$ imagens para teste, o método de validação consiste em dividir o conjunto de imagens em $10$ conjuntos, onde $1$ dessas $9$ partes são destinadas à validação, para análise dos hiperparâmetros e sua mudança, uma vez que são determinados de forma subjetiva para buscar o resultado mais satisfatório. O Cross-Validation foi implementado usando a biblioteca sci-kit, e portanto se assegura que o projeto não está sob overfit, a porcentagem de acertos com o conjunto de dados de validação foi de aproximadamente 82\%.